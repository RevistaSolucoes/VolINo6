
Este año el ganador del Premio Nobel de Fisica ha sido otorgado a
\emph{Isamu Akasaki}, \emph{Hiroshi Amano} y \emph{Shuji Nakamura} por
su desarrollo en LED's azules eficientes.

Hace ya años que el LED lo encontramos en nuestro entorno: pantallas
de los teléfonos inteligentes, tablets, ordenadores y televisores,
coches y bombillas de los hogares. Las principales caracterísitcas de
los LEDs es que son más duraderos, baratos, pequeños y eficientes con
hasta 300 lumen/watio a los 70 de los fluorescentes y 16 de las
bombillas tradicionales.

\section{La física del los LEDs}

Vamos a empezar por lo más interesante y complejo: entender un LED. Un LED es una unión de dos semiconductores tipo p-n. El tipo de semiconductor nos dice si los portadores de corriente serán negativos (electrones) o positivos (huecos sin electrón). Es importante entender que un hueco en el que falta un electrón se comporta como un electrón con carga positiva, aunque realmente no haya nada ahí.

Lo interesante de estas uniones es que forman diodos que solo dejan pasar la señal en una dirección. Esto permite que conviertan una señal alterna en contínua, por ejemplo. Los LEDs son unos diodos especiales que emiten luz cuando pasa corriente en la dirección permitida. Esta emisión de luz se produce por un salto de los electrones entre niveles de energía y deben cumplirse unas propiedas para que exista y podamos ver esa luz.

Gaps
Cada línea es un nivel de energía.

En concreto tienen que cumplirse dos condiciones sencillas, la primera de las cuales es que el mínimo de un nivel de energía se encuentre justo encima del máximo del nivel anterior. En la imagen vemos este efecto en el esquema de niveles. Estos semiconductores se llaman de “gap directo” (gap es la diferencia de energías) por razones que os imagináis.

La segunda condición es que el gap de energía entre un nivel y otro sea tal que el fotón resultante se emita tenga una frecuencia en el rango visible. Lo que dicho en cristiano significa que tenemos que hacer el gap del tamaño justo para poder ver la luz y que no sea infrarroja (gap pequeño) o ultravioleta (gap grande). Modificando el tamaño del gap podemos variar el color desde el rojo hasta el azul pasando por todos los del arcoiris.

LEDScreen

Y justo aquí, al final, llega el problema. Si aumentamos mucho el gap (por encima del color verde) es más fácil perder las propiedades de semiconductor y pasar a un aislante convencional como el cuarzo o el vidrio. Aquí es donde reside la dificultad de conseguir un LED azul: aumentar el gap lo suficiente manteniendo un semiconductor. Como veremos a continuación esto no es sencillo y costó mucho tiempo y dinero conseguirlo de una forma viable para la producción masiva.

Historia del LED azul

Ahora que ya sabemos dónde reside la dificultad de conseguir un LED azul veamos cómo fue la evolución histórica de este hito. Empezaremos con el primer LED, con emisión en rojo, que fue construido en 1962. Aunque ya antes se habían observado fenómenos similares en el infrarrojo. Desde entonces se mantuvo una carrera por conseguir el resto de colores. A pesar de tenerse los diseños desde los años 50 el LED azul aún se haría esperar varias décadas.

Entorno a 1970 la mejora en las técnicas de crecimiento de cristales permitió un gran avance en el desarrollo de nuestros queridos LEDs azules. En principio se intentaron basar en GaN (Nitruro de Galio) pero pronto se vio que esa técnica no conseguía una luminosidad suficiente. Es aquí donde podemos establecer la creación del primer LED azul, aunque no era usable y apenas se veía su luz.

Ganadores del Premio Nobel de Física 2014
Ganadores del Premio Nobel de Física 2014

Para los más exigentes podemos establecer 1989 como la fecha en que se consiguió el primer LED azul con una emisión razonablemente alta, aunque su eficiencia era del 0.03% Una vez más parecía que el LED azul no era viable para la producción masiva; hasta que en 1994 nuestros laureados obtuvieron por primera vez un LED azul de “alta” eficiencia utilizando técnicas modernas. Como semiconductor usaron InGaN/AlGaN y obtuvieron eficiencia alrededor de 2.7% (comparable al 4% de las bombillas incandescentes).

A día de hoy los LEDs corrientes que podemos comprar en cualquier tienda, muy baratos, tienen una eficiencia superior al 50% y presentan la mejor fuente de luz artificial que conocemos. El LED azul eficiente (muy importante esta última palabra) ha permitido, en primer lugar, completar las matrices RGB que usan hoy todas las pancartas LED del mundo así como obtener LEDs blancos.

Screen Shot 2014-10-07 at 19.34.36

Sí, blancos, no os había hablado de ellos porque me lo guardaba para el final. El LED blanco funciona como los fluorescentes de casa, pero mejor. Tenemos que bombardear con luz de alta energía (azul, ultravioleta…) un fosfato de tal forma que su reacción sea emitir luz en todo el espectro visible, dando lugar al color blanco. No es así como se hace el blanco en la pantalla de tu móvil (se enciende un pixel verde otro rojo y otro azul), pero sí es la forma de conseguir bombillas caseras o de iluminación de calles.

En resumen, y ya termino la chapa, los galardonados este año son más que merecidos ganadores del premio Nobel pues han permitido una revolución tecnológica equiparable a la del PC. En un primer momento puede parecer una elección frívola y sin fundamento, pero a mi juicio ya era hora de que lo obtuvieran. ¿Vosotros que pensáis? ¿Estáis de acuerdo o creéis que ha sido una decisión “porque no había nada mejor”? Como siempre, los comentarios son vuestros.




 Los tres recibirán el premio en
Estocolmo el 10 de diciembre.  La cita premio honra el trío de "la
invención de los diodos emisores de luz azul eficientes que ha
permitido brillante y fuentes de luz blanca de ahorro de energía". Los
LEDs ahora ubicuas se utilizan en una amplia arreglos de aplicaciones,
desde televisores a esterilizadores y no contienen mercurio tóxico que
se encuentra en las lámparas fluorescentes.  El blues de tres colores
Una fuente de luz blanca necesita LEDs que proporcionan luz roja,
verde y azul. El primer LED rojo fue creado en la década de 1950 y los
investigadores logró crear dispositivos que emiten luz en longitudes
de onda más cortas, llegando verde por la década de 1960. Sin embargo,
los investigadores se esforzaron por crear la luz azul.  En la década
de 1980 Akasaki y Amano de trabajo en la Universidad de Nagoya
Nakamura y que trabaja en el Nichia Corporation, centrado en el
nitruro de galio semiconductor compuesto (GaN), que podría ser ideal
para la creación de LEDs azules porque tenía una gran energía de
intervalo de banda correspondiente a la luz ultravioleta.  Había
muchos retos, sin embargo, en la fabricación de LEDs utilizables
basado en GaN. Un problema importante era cómo crear cristales de alta
calidad de GaN con buenas propiedades ópticas. Esto se resolvió de
forma independiente a finales de 1980 y principios de 1990 por Akasaki
y Amano y también por Nakamura. Ambos equipos utilizan metalorgánico
epitaxia en fase vapor (MOVPE) técnicas para depositar películas
delgadas de cristales de GaN de alta calidad sobre sustratos.
descubrimiento de dopaje Otro desafío aparentemente insuperable frente
a los investigadores fue la forma para dopar el GaN por lo que es un
semiconductor de tipo p, que es crucial para la creación de un
LED. Akasaki y Amano notaron que cuando GaN dopado con zinc se coloca
en un microscopio de electrones, emite mucha más luz. Esto sugiere que
la irradiación de electrones mejoró la p-dopaje - un efecto que más
tarde fue explicado por Nakamura.  Imagen de los LED azules En el azul
El siguiente paso para los dos equipos era utilizar su alta calidad, p
GaN dopado junto con otros semiconductores basados en GaN en
estructuras multicapa "heterounión". Nakamura fue capaz de crear el
primer azul de alto brillo LED en 1993.  Alabando los galardonados, el
presidente del comité Nobel de Física por Delsing dijo "Muchas de las
grandes empresas trató de [desarrollar LEDs azules] y que fracasó,
pero estos chicos persistió y finalmente lo lograron."  Hoy en día,
los LED de GaN se utilizan en pantallas de cristal líquido
retroiluminado en dispositivos que van desde teléfonos móviles a
pantallas de televisión. LEDs emisores de luz (UV) luz azul y
ultravioleta también se han utilizado en DVD, donde la longitud de
onda más corta de la luz permite densidades de almacenamiento de datos
más altas. Mirando hacia el futuro, los LED UV emisores podrían
utilizarse para crear sistemas de purificación de agua pero eficaces
básicos, ya que la luz UV puede destruir microorganismos.  Invención o
descubrimiento?  En los últimos 10 años se han producido otros tres
premios Nobel de física otorgados para el trabajo con un potencial
comercial importante: la magnetorresistencia gigante en 2007; fibra
óptica y dispositivos de transferencia de carga en 2009; y el grafeno
en 2010. Aunque la mayoría de premios están asociados con los
descubrimientos más esotéricas, como el bosón de Higgs, Alfred Nobel
decretó en su testamento que el premio también se podría dar para una
invención importante en la física.  "Alfred Nobel sería muy feliz por
este premio", dice Delsing. "[El LED azul] es realmente algo que la
mayoría de las personas se beneficiarán."  David Gross, del Instituto
Kavli de Física Teórica en la Universidad de California, Santa
Bárbara, quien compartió el premio Nobel de 2004 por su trabajo sobre
la libertad asintótica, se alegra de que en los últimos años tanto en
la investigación pura y aplicada están siendo reconocidos. Después de
dirigirse a una reunión en Trieste con motivo del 50 aniversario del
Centro Internacional de Física Teórica, donde se había hecho hincapié
en la importancia de la investigación azul-cielo, Gross dijo a Physics
World que "Cada cinco o seis años, el premio se concede a una
invención que ha conferido un gran beneficio para la humanidad, como
el transistor, la óptica del láser y la fibra. creo que la relación
existente es sólo de derecho ".  Akasaki nació en Chiran, Japón, en
1929. Se graduó de la Universidad de Kyoto en 1952 y recibió su
doctorado en 1964 en la Universidad de Nagoya.  Amano nació en
Hamamatsu, Japón, en 1960. Obtuvo su doctorado en 1989 en la
Universidad de Nagoya.  Nakamura nació en Ikata, Japón, en 1954. Se
graduó de la Universidad de Tokushima en 1977 con un título en
ingeniería electrónica y obtuvo una maestría en la misma materia dos
años más tarde. Luego se unió a la Nichia Corporation, una pequeña
empresa ubicada en Tokushima en la isla de Shikoku. Nakamura fue
galardonado con un doctorado en 1994 en la Universidad de Tokushima.


\newpage

%%% Local Variables: 
%%% mode: latex
%%% TeX-master: "novedades"
%%% End: 



