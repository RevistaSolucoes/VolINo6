\section*{Introducción}

En este artículo de final de año vamos a dar cierre al curso de Octave
que os hemos dado a lo largo de este primer volumen de la revista con 
una series de problemas que podemos resolver con \emph{Octave}. Espero leáis 
los problemas y antes de mirar la solución os déis un tiempo para hacerlo
por vosotros mismos. Es maravilloso encontrar la solución por uno
mismo. 

Durante el año hemos dado unas nociones básicas  de \emph{Octave} y puedes 
aumentar los conocimientos usando, por ejemplo, el siguiente tutorial
\url{http://en.wikibooks.org/wiki/Octave_Programming_Tutorial}.

\section{Esperanza de vida}
 
Según la organizacion mundial de la salud la definición de esperanza de vida 
es como sigue: años que un recién nacido puede esperar vivir si los patrones
de mortalidad por edades imperantes en el momento de su nacimiento siguieran
siendo los mismos a lo largo de toda su vida. Hacer un cálculo de esta cantidas 
es muy complicado y se tiene en cuenta factores como por ejemplo la medicina,
la higiene, las guerras, etc.

Lo que os proponemos es lo siguiente: dado la esperanza de vida de Ángola 
proporcionada por el banco mundial de datos (\url{http://datos.bancomundial.org/})
que tenemos datos desde 1980 a 2012, os proponemos resolváis el desafío A.

\begin{mybox}
\cappar A  ¿Cuál será la esperanza de vida de un ángolano en 2050 o en 2070 teniendo en cuenta
los datos dados?
\vspace{1.5cm}
\end{mybox}

\textcolor{blue}{\Large Nuestra Solución}
Hemos coleccionado los datos en un archivo \emph{Hopelife.csv}. Los vamos a cargar a Octave
y vamos a echarle un vistazo a estos. Para ello vamos a dibujarlos. Mostramos en la Tabla \ref{Tabla1}
el código ejecutadoa la izquierda y la salida o utput a la derecha.




%\vspace{3cm}
%\noindent
%\includegraphics[width=\textwidth]{pubmm2.png}

\newpage
%%% Local Variables: 
%%% mode: latex
%%% TeX-master: "informaticaeningenieria"
%%% End: 



