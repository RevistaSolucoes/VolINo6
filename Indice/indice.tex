\documentclass{scrartcl}

% Spanish 
\usepackage[portuguese]{babel}
\usepackage[utf8]{inputenc}

%Tamaño de la area de texto
\addtolength{\headheight}{-1cm}
\addtolength{\topmargin}{-1cm} 
\addtolength{\textwidth}{4.5cm}
\addtolength{\textheight}{8cm}
%\addtolength{\evensidemargin}{-3cm}
\addtolength{\oddsidemargin}{-2.5cm}
%Fonts
\usepackage{venturis2}
\usepackage[T1]{fontenc}

\usepackage[usenames,dvipsnames,svgnames,table]{xcolor}

\newcommand{\mc}[2]{\multicolumn{#1}{c}{#2}}
\definecolor{Gray}{gray}{0.85}
\definecolor{LightCyan}{rgb}{0.88,1,1}

\newcolumntype{a}{>{\columncolor{Gray}}c}
\newcolumntype{b}{>{\columncolor{white}}c}

\definecolor{redsol}{HTML}{ce161c}
\definecolor{bluesol}{HTML}{00467e}
\definecolor{graysol}{HTML}{58585a}
\definecolor{newgreen}{HTML}{4bb148}
\definecolor{newyel}{HTML}{d0b224}
\definecolor{newyell}{HTML}{FCD209}

\definecolor{sccol}{HTML}{FFD200}
\definecolor{scicol}{HTML}{4CBB47}
%\definecolor{sciiicol}{HTML}{BD3FC6}
\definecolor{sciiicol}{HTML}{F7AA02}

      \definecolor{csc}{HTML}{FFD200}
     \definecolor{cscc}{HTML}{DFC220}
    \definecolor{csccc}{HTML}{AFB240}
   \definecolor{cscccc}{HTML}{8EA260}
  \definecolor{csccccc}{HTML}{6EA080}
 \definecolor{cscccccc}{HTML}{4E90A0}
\definecolor{csccccccc}{HTML}{2D80C1}

\usepackage{tabularx} % in the preamble
\usepackage{graphicx}

\begin{document} 
{\fontsize{200}{200}\selectfont \hfill Indice}

\vspace{1cm}

\hfill\rotatebox{90}{
{\Huge\color{white}
\begin{tabularx}{0.72\textheight}{Xr}
{\color{redsol}\fontsize{150}{150}\selectfont Soluções}&\\
&\\
{\color{bluesol}\fontsize{50}{50}\selectfont Vol. I, No. 6}&\\
&\\
\rowcolor{csccccccc}
Nada é impossível:  María José Peláez Montalvo  & {\fontsize{50}{50}\selectfont 1}\\
\rowcolor{cscccccc}
Solução em ação: Luis Arantes Lemos de Azevedo &  {\fontsize{50}{50}\selectfont 3}\\
\rowcolor{csccccc}
Matemática em engenharia: Análise harmónica & \\
\rowcolor{csccccc}
clássica, Parte II & {\fontsize{50}{50}\selectfont 5}\\
\rowcolor{cscccc}
Informática en ingeniería: aplicaciones en \emph{Octave} & {\fontsize{50}{50}\selectfont 17}\\
\rowcolor{csccc}
Novedades: Nobel de Física 2014 & {\fontsize{50}{50}\selectfont 24}\\
\rowcolor{cscc}
Pensar jugando&  {\fontsize{50}{50}\selectfont 26}\\
\end{tabularx}}}
\end{document}
%%% Local Variables: 
%%% mode: latex
%%% TeX-master: t
%%% End: 


