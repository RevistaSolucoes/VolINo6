\begin{multicols}{2}

  \cappar En este número os damos la solución al desafío del anterior
  número y os dejamos un par de desafios nuevos para enfrentarnos en
  final de año. ¡Disfrútenlos!


\section*{Desafíos fín de año}
\subsection{Desafío Día de la Familia}
Proponga en su familia que cada uno de ellos piense en un número cualquiera de tres cifras
que no termine en cero, y les pide que cada uno ponga las cifras en orden
contrario. Hecho esto, deben restar del número mayor el menor y la
diferencia obtenida sumarla con ella misma pero con las cifras
escritas en orden contrario. Sin preguntar nada, adivina el número
a todos. ¿Qué ha pasado?
\subsection{Desafíos año nuevo}
\begin{enumerate}
\item Un comité se ha reunido 40 veces. Hubo 10 miembros en
  todas las reuniones. Ni una sola pareja se ha reunido en las
  reuniones dos veces.  Demuestre que como mínimo había 60 miembros en
  el comité.
\item Un día de frío, una persona mayor y un niño están al aire libre. Ambos van igualmente vestidos. ¿Cuál de los dos tiene más frío?
\end{enumerate}
{\bf Las soluciones en el próximo número.}
\section*{\textcolor{redsol}{Soluciones del número anterior}}
\subsection{El desafío: Monty Hall extendido}
Esta vez, vamos a daros la solución al problema de Monty Hall con
\textbf{cuatro} puertas, una de las cuales esconde un premio que
quieres ganar y cuando eliges por primera vez, el presentador te abre
dos puertas malas y te da a elegir entre tu puerta y la que queda sin
abrir.

Cuando eliges una puerta, tienes $1/4$ de probabilidad de acertar, y
por lo tanto, tienes $3/4$ de probabilidades de haber fallado
(evidentemente, esta probabilidad se reparte entre las tres puertas
que no se han elegido, cada una con $1/4$).

Cuando el presentador abre las dos puertas malas, tu probabilidad de
haber acertado no cambia en absoluto, sigue siendo $1/4$. Por lo
tanto, tampoco cambia la probabilidad de que hayas fallado, que sigue
siendo $3/4$. Por lo tanto, tenemos dos puertas. El premio está en la
tuya con probabilidad $1/4$ y debe estar en la otra con probabilidad
$3/4$, de modo que lo más recomendable sería cambiar de puerta.
%\end{multicols}

\begin{center}
\begin{tikzpicture}
\coordinate (star) at (0,-1);
\path (star) +(-50:7) coordinate (rhs) +(-130:7) coordinate (lhs);
\draw[brown!50!black,line width=5mm,line cap=round] (star) ++(-90:6.8) -- ++(0,-1) coordinate (base);
\node[scale=-1,trapezium,fill=black,minimum size=1cm] at (base) {};
\foreach \height/\colour in {%
  .2/blue,
  .4/yellow,
  .6/red,
  .8/orange,
  1/pink%
} {
  \draw[tinsel=\colour] ($(star)!\height!(lhs)$) to[bend right] ($(star)!\height!(rhs)$);
}
\path (star);
\pgfgetlastxy{\starx}{\stary}
\begin{scope}[xshift=\starx,yshift=\stary,yshift=-7cm]
\draw[color=green!50!black, l-system={rule set={S -> [+++G][---G]TS,  G -> +H[-G]L, H -> -G[+H]L, T -> TL, L -> [-FFF][+FFF]F}, step=4pt, angle=18, axiom=+++++SLFFF, order=11}] lindenmayer system -- cycle;
\end{scope}
\foreach \height/\colour in {%
  .1/pink,
  .3/red,
  .5/yellow,
  .7/blue,
  .9/orange%
} {
  \draw[tinsel=\colour] ($(star)!\height!(lhs)$) to[bend right] ($(star)!\height!(rhs)$);
}
\foreach \height in {.15,.35,...,1} {
  \draw[lights]  ($(star)!\height!(lhs)$) to[bend right] ($(star)!\height!(rhs)$);
}
\foreach \angle/\colour in {
  -50/red,
  -70/yellow,
  -90/blue,
  -110/pink,
  -130/purple%
} {
  \draw[baubles=\colour] (star) -- ++(\angle:7);
}
\node[star,star point ratio=2.5,fill=yellow,minimum size=1cm] at (star) {};
\end{tikzpicture}
\LARGE{\textcolor{red}{Feliz Natal}}
\end{center}


\end{multicols}
\newpage


%%% Local Variables:
%%% mode: latex
%%% TeX-master: "jugando"
%%% End:



