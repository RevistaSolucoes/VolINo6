%{\normalfont\initfamily \fontsize{12mm}{12mm}\selectfont D}
\begin{multicols}{2}
\cappar Estimados lectores de nuestra revista, en este último número de nuestro primer año en la revista me he permitido escribir este artículo para contaros con un poco detalle como se está haciendo la historia de esta revista. Me llamo María y soy la coordinadora de la revista. Hace poco más de un año la empresa Saema me ofreció colaborar en la creación de una revista científica--divulgativa donde tuviésemos la oportunidad de mostrar como las matemáticas son una herramienta hermosa en temas de ingeniería, especialmente en temas de energía y aguas que es donde son especialistas la empresa Saema. Cuando yo comencé ya estaba nuestro compañero Bartlomiej Skorulski, también doctor en matemáticas, montando el diseño de la revista y escribiendo los primeros artículos sobre la catenaria. Estábamos emocionados de participar en un proyecto donde nuestro principal objetivo era daros, a vosotros, nuestros lectores, una buena calidad de información.

 Como sabéis a día de hoy nuestra revista tiene varias secciones. La primera sección la llamamos Nada es imposible y hoy me he colado yo en ella. Esta sección quería que, vosotros y a veces nosotros, hablásemos de nuestra experiencia con las ciencias, de como vemos las ciencias en Ángola y sobre todo de hacer ver que no hay nada imposible que no se pueda hacer con fuerza e ilusión. La segunda sección titulada Solución en acción estaba destinada a gente que hace posible que hoy Saema esté aquí. Hemos recorrido por varias experiencias de trabajadores de Saema y de amigos de Saema contándonos su día a dia y mandando mensajes al pueblo angolano. La sección de Matemáticas en ingeniería es el corazón de la revista. Es propiamente lo que más la define e intentamos explicaros y acercaros a las matemáticas escritas entre postes de luz, entre tuberías y entre ondas de una buena música como la de Bonga. La sección de Informática en Ingeniería la pensamos porque cada vez es más importante en estas décadas tener algunos conocimientos de informática que te den chance para poder crear aquellas cosas que pueden rondar en tu cabeza. La programación está inmersa en el día a día y hacerte partícipe de ella para entender mejor el mundo creemos es importante. Este primer año nos hemos dedicado a un lenguaje funcional y muy potente que es Octave. Espero lo hayas disfrutado y hayas podido seguir las diferentes prácticas. En este último número del año te damos varios problemas pra desarrollar con Octave. Disfrútalos. La sección de Novedades era contaros algunas curiosidades que hemos pensado eran dignas de dedicarles un artículo para compartir con todos vosotros y por último mencionar la sección de Pensar jugando donde nos permitíamos salir al patio para jugar imaginativamente. 

Como sabéis Saema promociona un grupo de ajedrez y hemos querido dedicar algunas informaciones de ajedrez para apoyar algrupo y para haceros un guiño con este apasionante mundo delm ajedrez que es tan popular en todo el mundo.


Muchos de vosotros se que sois seguidores de nuestro facebook de la revista. ha sido divertidísimo colgar un artículo o una foto para vosotros que habéis compartido y visto mucho de vosotros y de vuestros amigos. Gracias por compartirnos y espero sigan compartiéndonos en los medios sociales.

La creación de esta revista lleva varias etapas, una primera etapa de discusión, de selección de temas por el equipo. Una etapa de estudio, de búsqueda de información y de creación. Una tercera etapa de revisión y discusión. Una cuarta etapa de maquetación. Una etapa de revisión final y llega el momento culminante: la impresión de la revista. En paralelo vamos subiendo los contenidos al formato digital donde os damos más informaciones, com son aplicaciones web diseñada por vosotros para que podáis jugar con cols conceptos o usarlas para enseñarlas a otros y los códigos de los porgramsa y gráficos expuestos en la revista. Y la etapa final es que llegue a vosotros y la recibáis, como me han contado, con una sonrisa obrigada.

A futuro de la revista me encantaría que nos escribiéseis más, que nos contéis de que os encantaría os hablásemos, que temas os parece de interés, que inquietudes tenéis y si conocéis novedades que mencionar mandárnoslo. Estamos abiertos y felices de poder conoceros e intentad daros mediante nuestra revista una infinita satisfacción de amor a la ciencia.

Aprovecho para desearos un feliz 2015 en nombre de todo mi equipo. Sigan soñando y recordando que una gran aportación no es sino una suma de pequeñas aportaciones. 

\begin{wrapfigure}{r}{0.40\textwidth} 
  \vspace{-21pt}
  \begin{figurebox}
   \vspace{20pt}
    \centering
%    \includegraphics[height=0.35\textheight]{Enrique.jpg}\\
    María José Peláez Montalvo\\ 
    {\small Coordinadora Revista Soluçoes}\\
    \vspace{1pt}
    %\vspace{0.1\textheight}
  \end{figurebox}
 \vspace{-20pt}
\end{wrapfigure}

\end{multicols}

\newpage
%%% Local Variables: 
%%% mode: latex
%%% TeX-master: "nadaesimposible"
%%% End: 


